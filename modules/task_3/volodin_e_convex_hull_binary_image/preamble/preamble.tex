\documentclass[a4paper, 12pt]{extarticle}

% Пакеты для поддержки языков
\usepackage[english, russian]{babel}

% Настройка кодировок
\usepackage[T2A]{fontenc}
\usepackage[utf8]{inputenc}

% Настройка шрифтов
\usepackage{fontspec}
\setmainfont[Ligatures=TeX]{Times New Roman} % Шрифт основного текста
\setmonofont{Consolas} % Шрифт кода

% Настройка отступов от краев страницы
\usepackage[left=3cm, right=1.5cm, top=2cm, bottom=2cm]{geometry}

\usepackage{titleps} % Колонтитулы
\usepackage{subfig} % Для подписей к рисункам и таблицам
\usepackage{graphicx} % Для вставки картинок
\graphicspath{{./img}} % Путь до папки с изображениями

% Пакеты для отрисовки графиков
\usepackage{tikz}
\usetikzlibrary{arrows, positioning, shadows}

\usepackage{stmaryrd} % Стрелки в формулах
\usepackage{indentfirst} % Красная строка после заголовка

% Пакеты для работы с таблицами
\usepackage{hhline} % Улучшенные горизонтальные линии в таблицах
\usepackage{multirow} % Ячейки в несколько строчек в таблицах
\usepackage{longtable} % Многостраничные таблицы
\usepackage{paralist, array} % Список внутри таблицы

\usepackage[normalem]{ulem} % Зачеркнутый текст
\usepackage{upgreek, tipa} % Красивые греческие буквы

% Математические пакеты
\usepackage{amsmath, amsfonts, amssymb, amsthm, mathtools}

\usepackage{nicematrix} % Особые матрицы (Гауссовские)

\linespread{1.5} % Межстрочный интервал
\setlength{\parindent}{1.25cm} % Табуляция
\setlength{\parskip}{0cm}

% Пакет для красивого выделения кода
\usepackage{minted}
\setminted{fontsize=\footnotesize}

% Добавляем гипертекстовое оглавление в PDF
\usepackage[
bookmarks=true, colorlinks=true, unicode=true,
urlcolor=black, linkcolor=black, anchorcolor=black,
citecolor=black, menucolor=black, filecolor=black,
]{hyperref}

% Убрать переносы слов
\tolerance=1
\emergencystretch=\maxdimen
\hyphenpenalty=10000
\hbadness=10000

\newpagestyle{main}{
	% Верхний колонтитул
	\setheadrule{0cm} % Размер линии отделяющей колонтитул от страницы
	\sethead{}{}{} % Содержание {слева}{по центру}{справа}
	% Нижний колонтитул
	\setfootrule{0cm} % Размер линии отделяющей колонтитул от страницы
	\setfoot{}{\thepage}{} % Содержание {слева}{по центру}{справа}
}
\pagestyle{main}

% Новые команды
\newcommand{\n}{\par}
\newcommand{\percent}{\mathbin{\%}}

% Заменяем Рис. на Рисунок
\addto\captionsrussian{\renewcommand{\figurename}{Рисунок}}

% Изменение формата подписей
% Стиль номера таблицы/рисунка #1-Таблица/Рисунок. #2-номер
\DeclareCaptionLabelFormat{custom}
{
	#1 #2.
}
% Стиль разделителя номера таблицы/рисунка и названия таблицы/рисунка
\DeclareCaptionLabelSeparator{custom}{}
% Стиль формата #1-номер таблицы/рисунка #2-разделитель #3-название
\DeclareCaptionFormat{custom}
{
	#1 #3
}

\captionsetup
{
	format=custom,
	labelsep=custom,
	labelformat=custom
}
