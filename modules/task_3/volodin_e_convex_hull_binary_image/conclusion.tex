\section{Заключение}
В данной лабораторной работе мы реализовали алгоритм построения минимальной выпуклой оболочки для компонент бинарного изображения с использованием последовательного и параллельного подходов.\n
Наша основная цель была создать эффективный параллельный алгоритм, который бы работал быстрее, чем последовательный, при использовании нескольких процессов.\n
Мы провели ряд тестов, чтобы проверить корректность наших алгоритмов и сравнить их производительность. Тесты показали, что наши алгоритмы правильно вычисляют оболочку для различных входных данных. Кроме того, мы получили положительные результаты вычислительных экспериментов, которые подтверждают, что наш параллельный алгоритм дает значительное ускорение по сравнению с последовательным.\n
Таким образом, мы успешно выполнили поставленные задачи и достигли желаемого результата.