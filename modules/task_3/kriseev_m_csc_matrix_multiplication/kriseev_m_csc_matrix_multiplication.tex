\documentclass[a4paper,12pt]{article}
\usepackage[T2A]{fontenc}
\usepackage[utf8]{inputenc}
\usepackage[english,russian]{babel}
\usepackage{listings}
\usepackage[colorlinks=false]{hyperref}
\usepackage{indentfirst}
\usepackage{multirow}
\usepackage{enumitem}
\usepackage{titlesec}
\usepackage{secdot}
\usepackage{tocloft}
\titlelabel{\thetitle. \quad}
\usepackage{geometry}
\geometry{left=2.5cm}
\geometry{right=1.5cm}
\geometry{top=2cm}
\geometry{bottom=2cm}

\titlespacing*{\section}{0em}{0ex}{1em}

\usepackage{color}
\definecolor{dkgreen}{rgb}{0,0.6,0}
\definecolor{gray}{rgb}{0.5,0.5,0.5}
\definecolor{mauve}{rgb}{0.58,0, 0.82}

\lstset{
language=C++,
aboveskip=3mm,
xleftmargin=10mm,
belowskip=3mm,
showstringspaces=false,
columns=flexible,
basicstyle={\ttfamily},
numbers=none,
numberstyle=\small\color{gray},
keywordstyle=\color{blue},
commentstyle=\color{dkgreen},
stringstyle=\color{mauve},
breaklines=true,
breakatwhitespace=true,
tabsize=3
}

\renewcommand{\thesection}{\arabic{section}}
\renewcommand{\cftsecleader}{\cftdotfill{\cftdotsep}}

\begin{document}
\begin{titlepage}
 \begin{center}
{\small МИНИСТЕРСТВО НАУКИ И ВЫСШЕГО ОБРАЗОВАНИЯ РОССИЙСКОЙ ФЕДЕРАЦИИ\\
 Федеральное государственное автономное образовательное учреждение
высшего образования
}
\textbf{<<Национальный исследовательский
Нижегородский государственный университет им. Н.И. Лобачевского>>\\
(ННГУ)\\}

\vspace*{1cm}

\textbf{Институт информационных технологий, математики и механики}

\vspace*{3cm}

\textbf{\LargeОТЧЁТ}

по лабораторной работе

\vspace{1em}

\textbf{\large<<Умножение разреженных матриц в формате CCS>>}

\end{center}

\vspace{2cm}

\begin{flushright}
\parbox{8.5cm}{
\textbf{Выполнил:}

студент группы 3821Б1ПР1\\
Крисеев М.А.

\textbf{Преподаватель:}

доцент кафедры высокопроизводительных вычислений и системного программирования\\
кандидат технических наук\\
Сысоев А.В.
}
\end{flushright}

\flushbottom

\vfill

\begin{center}
 Нижний Новгород

 2023
\end{center}

\end{titlepage}

\tableofcontents

\newpage

\section{Введение}
\par
При решении задач, связанных с использованием матриц, зачастую приходится работать с \textit{разреженными матрицами}.
Такие матрицы содержат в себе большое колисество нулевых элементов, из-за чего хранить все эти элементы -- неэффективно.

Проблему хранения разреженных матриц можно решить множеством способов. Один из них -- хранение ненулевых элементов по столбцам, CCS, Compressed column storage (также используется название CSC -- Compressed sparse column).

Для представления разреженной матрицы размера $M$x$N$ в формате CCS используется три массива:

\begin{itemize}
\item $values$ -- содержит все ненулевые значения элементов матрицы,
\item $rowIndices$ -- содержит номера строк, на которых находятся соответствующие элементы $values$,
\item $columnPointers$, $C_p$ -- массив размера $N+1$, каждый элемент которого задаёт индекс элемента из предыдущих массивов, с которого начинается соответствующий столбец матрицы и которым заканчивается предыдущий столбец. $C_p[0]$ всегда равен 0, $C_p[N]$ всегда равен размеру $values$.

Несмотря на б\'{о}льшую эффективность хранения разреженных матриц в формате CCS, некоторые типовые задачи, такие как умножение матриц, при этом приходится реализовывать по-другому. Именно умножение матриц и рассмотрено в данной лабораторной работе.

\end{itemize}
\newpage
\section{Основная часть}
\subsection{Постановка задачи}
\textbf{Задача:} реализовать алгоритм умножения двух разреженных матриц, сохранённых в формате CCS. Реализовать параллельную версию данного алгоритма. Написать программу, испольняющую данные алгоритмы, для параллельной версии использовать библиотеку MPI. Провести эксперименты и замерить производительность данных алгоритмов.
\subsection{Описание алгоритма}

Умножение матриц представляет собой множество скалярных произведений строк первой матрицы на столбцы второй. Поскольку обе матрицы хранятся в столбцовом формате CCS - производить итерации по строкам становится неудобно, поэтому первое, что целесообразно сделать -- \textit{транспонировать первую матрицу}.

Для этого создаём новую матрицу, массивы $rowIndices$ и $values$ оставляем пустыми, в массив $columnPointers$ добавляем 0.

Находим наибольший индекс строки из массива $rowIndices$ исходной матрицы.
Проходим циклом по строкам до этого индекса, для каждой строки проходим по всему массиву $rowIndices$ исходной матрицы, и если индекс $i$ строки равен индексу элемента $j$ из $rowIndices$, то добавляем $values[j]$ в конец массива $values$ новой матрицы. Находим, какому столбцу в исходной матрице принадлежит элемент c индексом $j$ и добавляем индекс этого столбца в конец $rowIndices$ новой матрицы. После завершения цикла по $rowIndices$ длину $rowIndices$ новой матрицы добавляем в конец её $columnPointers$.

В результате получаем транспонированную матрицу, записанную в формате CCS.

Вторым шагом является непосредственно умножение двух матриц.

В результате умножения должна получиться матрица, содержащая столько же строк, сколько их в первой (не транспонированной) матрице, и столько же столбцов, сколько их во второй матрице. Это можно использовать, чтобы организовать цикл по всем будущим элементам результирующей матрицы, внешний цикл по её столбцам, вложенный - по строкам.

Для каждого элемента можно использовать следующий алгоритм скалярного произведения разреженных столбцов (предполагая, что число строк равно $N$):
\begin{enumerate}[label*=\arabic*.]
 \item Инициализируем результат умножения нулём,
 \item Заполняем массив $L$ из $N$ элементов значениями $-1$,
 \item Проходим по всем ненулевым элементам первого столбца:
 \begin{enumerate}[label*=\arabic*.]
    \item Индекс $k$ текущего элемента в $rowIndices$ записываем в $L$ по индексу, равному $rowIndices[k]$,
\end{enumerate}
\item Проходим по всем элементам второго столбца (номер элемента: $m$):
\begin{enumerate}[label*=\arabic*.]
 \item Если в $L$ по индексу, равному $rowIndices[m]$, лежит значение, не равное $-1$, то
 к результату прибавляем значение текущего элемента, умноженное на значение $a.values[L[rowIndices[m]]]$, где $a$ -- первая матрица
\end{enumerate}
\item Если результат не равен нулю, то добавляем его в конец $values$ результирующей матрицы.
Значение строки (номер итерации внутреннего цикла) добавляем в конец $rowIndices$ результирующей матрицы.
\end{enumerate}

После каждого завершения работы вложенного цикла в массив $columnPointers$ добавляем в конец размер $rowIndices$.

В конце концов, получаем новую разреженную матрицу, являющуюся результатом произведения двух исходных.

\subsection{Описание схемы распараллеливания}

Поскольку все итерации по элементам результирующей матрицы независимы и используют только неизменяющиеся значения элементов исходных матриц, схема распараллеливания достаточно тривиальна.

Каждому процессу распределяется свой набор столбцов, которые ему необходимо вычислить. После этого все производят вычисления и результат собирается на главном потоке в готовую матрицу.

Несмотря на тривиальность распараллеливания самого процесса умножения, собрать готовую матрицу достаточно сложно.

Чтобы упростить данную задачу, каждому процессу выделяем последовательный набор столбцов для вычисления. Таким образом, при слиянии двух матриц, полученных от них, ненулевые столбцы одной всегда будут расположены после столбцов другой.

Находим матрицу, у которой ненулевые столбцы начинаются раньше. К её массиву $rowIndices$ добавляем все элементы $rowIndices$ другой матрицы. Аналогично поступаем с их массивами $values$. Затем складываем поэлементно значения их массивов $columnPointers$. Последовательно сливая матрицы, полученные от процессов, получаем искомую.

\subsection{Описание реализации на MPI}
Для реализации параллельного алгоритма была использована надстройка из boost над MPI.

Чтобы было удобнее работать с матрицами, три массива и две целочисленных переменных, обозначающих размеры матрицы, были объединены в один класс. Также в этот класс добавлены методы $findElementColumn(size\_t\ index)$ для поиска столбца, в котором находится элемент, и $transpose()$, возвращающий транспонированную копию данной матрицы. Была добавлена функция сериализации данного класса средствами boost.

При вызове функции параллельного алгоритма на нескольких процессах предполагается, что входные данные у всех одинаковые. Дополнительные проверки этого отсутствуют.

Количество столбцов на каждый процесс вычисляется на всех процессах. После завершения вычислений каждым процессом, вызывается обёртка функции MPI\_All\_reduce boost::mpi::all\_reduce. В качестве операции туда передаётся структура с перегруженным оператором <<()>>, который выполняет слияние двух матриц.

После этого готовый результат возвращается каждым процессом.

\subsection{Результаты экспериментов}
В качестве метрики производительности я использовал время выполнения. Хотя это и не является достаточно объективной метрикой и зависит от множества параметров, при запуске кода на одной и той же машине можно получить некоторое общее представление о быстродействии алгоритма и ускорении (или замедлении) параллельной версии.

Для измерения времени использована обёртка для boost над встроенным таймером MPI. Это даёт основания считать измерения корректными.

Все измерения проводились на ноутбуке. Процессор: AMD Ryzen 3 5300U. ОС: Arch linux.

В результате серии экспериментов получились следующие результаты:

\begin{tabular}{|p{3.5cm}|p{3cm}|p{3cm}|p{3cm}|}
 \hline
 & \multicolumn{3}{|c|}{Время работы}\\
 \hline
 Размер матрицы & 1 процесс & 2 процесса & 4 процесса\\
 \hline
 10x10 & $4.191*10^{-3}$ мс & $7.717*10^{-2}$ мс & $0.133$ мс\\
 \hline
 100x100 & $1.053$ мс & $1.756$ мс & $2.479$ мс\\
 \hline
 1000x1000 & $900$ мс & $652$ мс & $697$ мс\\
 \hline
\end{tabular}
\newpage
\section{Выводы}
В ходе экспериментов было выявлено, что на малых размерах умножаемых матриц последовательный алгоритм значительно обгоняет параллельный, однако с увеличением размеров эта разница уменьшается и в некоторый момент уже параллельный алгоритм начинает работать быстрее.

Также можно заметить, что скорость работы не увеличивается с увеличением числа процессов, а наоборот падает или остаётся на том же уровне. Это может свидетельствовать о некоторой неэффективности алгоритма.

Главной проблемой и <<бутылочным горлышком>> данной реализации алгоритма, на мой взгляд, является функция слияния. Хотя у меня получилось значительно её упростить, судя по всему, её выполнение всё еще занимает много времени.

\newpage
\section{Заключение}
В результате выполнения данной работы у меня получилось реализовать достаточно эффективный последовательный алгоритм умножения разреженных матриц, хранящихся в формате CCS, а также сделать параллельную версию данного алгоритма, которая менее эффективна на малых размерах матриц, но обгоняет последовательную на больших.
\newpage
\section{Литература}
\begin{enumerate}
\item Sparse matrix // Wikipedia. -- Режим доступа: https://en.wikipedia.org/wiki/Sparse\_matrix
\item Кустикова В.Д., Мееров И.Б., Сысоев А.В. Параллельные численные методы. Лабораторная работа:
Умножение разреженной матрицы на вектор // Сайт Центра суперкомпьютерных технологий ННГУ -- Режим доступа: http://www.hpcc.unn.ru/file.php?id=492

\end{enumerate}

\newpage
\section*{Приложение A}
\addcontentsline{toc}{section}{Приложение A}
Код заголовочного файла проекта:
\begin{lstlisting}[language=C++]
#ifndef CSC_MATRIX_MULTIPLICATION_H_
#define CSC_MATRIX_MULTIPLICATION_H_

#include <vector>
#include <boost/serialization/vector.hpp>

struct CscMatrix {
 public:
    std::vector<double> values;
    std::vector<size_t> rowIndices;
    std::vector<size_t> columnPointers;
    size_t rows;
    size_t cols;

    CscMatrix(size_t rows = 0, size_t cols = 0,
              const std::vector<double> &values = {},
              const std::vector<size_t> &rowIndices = {},
              const std::vector<size_t> &columnPointers = {0});
    CscMatrix(const CscMatrix &c);

    bool operator==(const CscMatrix &b) const;

    CscMatrix transpose() const;

    void print() const;

    void setElement(size_t column, size_t row, double value);

    size_t findElementColumn(size_t index) const;
};

CscMatrix multiplyCscMatricesSequential(const CscMatrix &a, const CscMatrix &b);
CscMatrix multiplyCscMatricesParallel(const CscMatrix &a, const CscMatrix &b);

struct MergeMatrices {
    CscMatrix operator()(const CscMatrix &a, const CscMatrix &b);
};

namespace boost {
namespace serialization {
template <class Archive>
void serialize(Archive &ar, CscMatrix &m,
               const unsigned int version) {
    ar & m.rows;
    ar & m.cols;
    ar & m.values;
    ar & m.rowIndices;
    ar & m.columnPointers;
}
}  // namespace serialization
}  // namespace boost

#endif  // CSC_MATRIX_MULTIPLICATION_H_

\end{lstlisting}
\newpage
\section*{Приложение B}
\addcontentsline{toc}{section}{Приложение B}
Код файла с реализацией функций
\begin{lstlisting}[language=C++]
#include "csc_matrix_multiplication.h"
#include <algorithm>
#include <iostream>
#include <limits>
#include <boost/mpi/communicator.hpp>
#include <boost/mpi/collectives.hpp>
#include <boost/serialization/vector.hpp>

void appendMatrix(CscMatrix &a, const CscMatrix &b) {
    a.rowIndices.reserve(a.rowIndices.size() + b.rowIndices.size());
    a.values.reserve(a.values.size() + b.values.size());

    for (int i = 0; i < b.rowIndices.size(); ++i) {
        a.rowIndices.push_back(b.rowIndices[i]);
        a.values.push_back(b.values[i]);
    }
    for (int i = 1; i < b.columnPointers.size(); ++i) {
        a.columnPointers[i] += b.columnPointers[i];
    }
}

CscMatrix::CscMatrix(size_t rows, size_t cols,
                     const std::vector<double> &values,
                     const std::vector<size_t> &rowIndices,
                     const std::vector<size_t> &columnPointers)
    : rows(rows),
      cols(cols),
      values(values),
      rowIndices(rowIndices),
      columnPointers(columnPointers) {}
CscMatrix::CscMatrix(const CscMatrix &c)
    : rows(c.rows),
      cols(c.cols),
      values(c.values),
      rowIndices(c.rowIndices),
      columnPointers(c.columnPointers) {}
bool CscMatrix::operator==(const CscMatrix &b) const {
    if (b.rows != rows || b.cols != cols ||
        b.rowIndices.size() != rowIndices.size()) {
        return false;
    }
    for (size_t i = 0; i < b.rowIndices.size(); ++i) {
        if (std::abs(values[i] - b.values[i]) >
            std::numeric_limits<double>::epsilon()) {
            return false;
        }
        if (rowIndices[i] != b.rowIndices[i]) {
            return false;
        }
    }
    for (size_t c = 0; c < b.columnPointers.size(); ++c) {
        if (columnPointers[c] != b.columnPointers[c]) {
            return false;
        }
    }
    return true;
}

CscMatrix CscMatrix::transpose() const {
    CscMatrix n = {this->cols, this->rows};
    if (this->rowIndices.empty()) {
        n.columnPointers.push_back(0);
        return n;
    }
    size_t maxRow =
        *std::max_element(this->rowIndices.begin(), this->rowIndices.end());
    for (size_t i = 0; i <= maxRow; ++i) {
        for (size_t j = 0; j < this->rowIndices.size(); ++j) {
            if (this->rowIndices[j] == i) {
                n.values.push_back(this->values[j]);
                n.rowIndices.push_back(this->findElementColumn(j));
            }
        }
        n.columnPointers.push_back(n.rowIndices.size());
    }
    return n;
}

void CscMatrix::print() const {
    std::cout << "v: ";
    for (auto v : values) {
        std::cout << v << " ";
    }
    std::cout << "r: ";
    for (auto v : rowIndices) {
        std::cout << v << " ";
    }
    std::cout << "c: ";
    for (auto v : columnPointers) {
        std::cout << v << " ";
    }
    std::cout << "\n";
}
void CscMatrix::setElement(size_t column, size_t row, double value) {
    size_t beginCol = columnPointers[column];
    size_t endCol = columnPointers[column + 1];
    size_t insertIndex = std::lower_bound(rowIndices.begin() + beginCol,
                                          rowIndices.begin() + endCol, row) -
                         rowIndices.begin();
    if (!rowIndices.empty() && row == rowIndices[insertIndex]) {
        values[insertIndex] = value;
        return;
    }

    values.insert(values.begin() + insertIndex, value);
    rowIndices.insert(rowIndices.begin() + insertIndex, row);

    for (size_t c = column + 1; c < columnPointers.size(); ++c) {
        columnPointers[c]++;
    }
}
size_t CscMatrix::findElementColumn(size_t index) const {
    for (size_t i = 0; i < columnPointers.size() - 1; i++) {
        if (index >= columnPointers[i] && index < columnPointers[i + 1]) {
            return i;
        }
    }
    return columnPointers.size();
}

CscMatrix multiplyCscMatricesParallel(const CscMatrix &a, const CscMatrix &b) {
    boost::mpi::communicator world;
    if (world.size() == 1) {
        return multiplyCscMatricesSequential(a, b);
    }
    if (a.cols != b.rows) {
        throw std::invalid_argument("b");
    }

    CscMatrix a_t(a.cols, a.rows);
    CscMatrix b_c(b);

    a_t = a.transpose();
    CscMatrix res = {a_t.cols, b.cols, {}, {}, {0}};

    std::vector<size_t> tempCol(a_t.rows, -1);
    std::vector<int> j_sizes(world.size(), res.cols / world.size());
    for (size_t k = 0; k < res.cols % world.size(); ++k) {
        j_sizes[k]++;
    }
    size_t myFirstIndex = 0;
    for (int i = 0; i < world.rank(); ++i) {
        myFirstIndex += j_sizes[i];
    }
    size_t myLastIndex = myFirstIndex + j_sizes[world.rank()];
    for (size_t j = 0; j < res.cols; ++j) {
        if (j >= myFirstIndex && j < myLastIndex) {
            for (size_t i = 0; i < res.rows; ++i) {
                double dot = 0;
                size_t colSizeA =
                    a_t.columnPointers[i + 1] - a_t.columnPointers[i];
                size_t colSizeB =
                    b_c.columnPointers[j + 1] - b_c.columnPointers[j];
                size_t beginColA = a_t.columnPointers[i];
                size_t beginColB = b_c.columnPointers[j];
                tempCol.assign(tempCol.size(), -1);
                for (size_t k = 0; k < colSizeA; ++k) {
                    size_t l = a_t.rowIndices[beginColA + k];
                    tempCol[l] = k;
                }
                for (size_t k = 0; k < colSizeB; ++k) {
                    size_t l = b_c.rowIndices[beginColB + k];
                    if (tempCol[l] == -1) {
                        continue;
                    }
                    dot += a_t.values[beginColA + tempCol[l]] *
                           b_c.values[beginColB + k];
                }
                if (dot != 0) {
                    res.values.push_back(dot);
                    res.rowIndices.push_back(i);
                }
            }
        }
        res.columnPointers.push_back(res.values.size());
    }
    CscMatrix finalRes(res.rows, res.cols);
    boost::mpi::all_reduce(world, res, finalRes, MergeMatrices());
    return finalRes;
}

CscMatrix MergeMatrices::operator()(const CscMatrix &a, const CscMatrix &b) {
    size_t aFirstCol = 0;
    for (size_t l = 0; l < a.columnPointers.size() - 1; ++l) {
        if (a.columnPointers[l] != a.columnPointers[l + 1]) {
            aFirstCol = l;
            break;
        }
    }
    size_t bFirstCol = 0;
    for (size_t l = 0; l < b.columnPointers.size() - 1; ++l) {
        if (b.columnPointers[l] != b.columnPointers[l + 1]) {
            bFirstCol = l;
            break;
        }
    }

    if (bFirstCol > aFirstCol) {
        CscMatrix res(a);
        appendMatrix(res, b);
        return res;
    } else {
        CscMatrix res(b);
        appendMatrix(res, a);
        return res;
    }
}

CscMatrix multiplyCscMatricesSequential(const CscMatrix &a,
                                        const CscMatrix &b) {
    if (a.cols != b.rows) {
        throw std::invalid_argument("b");
    }
    CscMatrix a_t = a.transpose();
    CscMatrix res = {a.rows, b.cols, {}, {}, {0}};
    std::vector<size_t> tempCol(a.cols, -1);
    for (size_t j = 0; j < res.cols; ++j) {
        for (size_t i = 0; i < res.rows; ++i) {
            double dot = 0;
            size_t colSizeA = a_t.columnPointers[i + 1] - a_t.columnPointers[i];
            size_t colSizeB = b.columnPointers[j + 1] - b.columnPointers[j];
            size_t beginColA = a_t.columnPointers[i];
            size_t beginColB = b.columnPointers[j];
            tempCol.assign(tempCol.size(), -1);

            for (size_t k = 0; k < colSizeA; ++k) {
                size_t l = a_t.rowIndices[beginColA + k];
                tempCol[l] = k;
            }
            for (size_t k = 0; k < colSizeB; ++k) {
                size_t l = b.rowIndices[beginColB + k];
                if (tempCol[l] == -1) {
                    continue;
                }
                dot += a_t.values[beginColA + tempCol[l]] *
                       b.values[beginColB + k];
            }

            if (dot != 0) {
                res.values.push_back(dot);
                res.rowIndices.push_back(i);
            }
        }
        res.columnPointers.push_back(res.values.size());
    }
    return res;
}

\end{lstlisting}

\end{document}
