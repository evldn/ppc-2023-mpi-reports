\documentclass{report}

\usepackage[warn]{mathtext}
\usepackage[T2A]{fontenc}
\usepackage[utf8]{luainputenc}
\usepackage[english, russian]{babel}
\usepackage[pdfpagemode=UseNone,colorlinks,allcolors=black]{hyperref}
\usepackage{tempora}
\usepackage[12pt]{extsizes}
\usepackage{listings}
\usepackage{color}
\usepackage{geometry}
\usepackage{enumitem}
\usepackage{multirow}
\usepackage{graphicx}
\usepackage{indentfirst}
\usepackage{amsmath}
\usepackage{soul}

\sethlcolor{gray}

\geometry{a4paper,top=2cm,bottom=2cm,left=2.5cm,right=1.5cm}
\setlength{\parskip}{0.5cm}
\setlist{nolistsep, itemsep=0.3cm,parsep=0pt}

\usepackage{listings}
\lstset{language=C++,
        basicstyle=\footnotesize,
		keywordstyle=\color{blue}\ttfamily,
		stringstyle=\color{red}\ttfamily,
		commentstyle=\color{green}\ttfamily,
		morecomment=[l][\color{red}]{\#}, 
		tabsize=4,
		breaklines=true,
  		breakatwhitespace=true,
  		title=\lstname,       
}

\makeatletter
\renewcommand\@biblabel[1]{#1.\hfil}
\makeatother

\begin{document}

\begin{titlepage}

\begin{center}
Министерство науки и высшего образования Российской Федерации
\end{center}

\begin{center}
Федеральное государственное автономное образовательное учреждение высшего образования \\
Национальный исследовательский Нижегородский государственный университет им. Н.И. Лобачевского
\end{center}

\begin{center}
Институт информационных технологий, математики и механики
\end{center}

\vspace{4em}

\begin{center}
\textbf{\LargeОтчет по лабораторной работе} \\
\end{center}
\begin{center}
\textbf{\Large<<Поразрядная сортировка для целых чисел с четно-нечетным слиянием Бэтчера>>} \\
\end{center}

\vspace{4em}

\newbox{\lbox}
\savebox{\lbox}{\hbox{text}}
\newlength{\maxl}
\setlength{\maxl}{\wd\lbox}
\hfill\parbox{7cm}{
\hspace*{5cm}\hspace*{-5cm}\textbf{Выполнил:} \\ студент группы 3821Б1ПР2\\Дерун А. А.\\
\\
\hspace*{5cm}\hspace*{-5cm}\textbf{Проверил:}\\ младший научный сотрудник\\Нестеров А. Ю.\\
}
\vspace{\fill}

\begin{center} Нижний Новгород \\ 2023 \end{center}

\end{titlepage}

\setcounter{page}{2}

% Содержание
\tableofcontents
\newpage

% Введение
\section*{Введение}
\addcontentsline{toc}{section}{Введение}
\par Сортировка данных является одной из ключевых операций в области обработки и анализа информации. В современном мире, где объемы данных растут в геометрической прогрессии, умение эффективно сортировать большие массивы данных является неотъемлемой частью множества задач и приложений. От электронной коммерции до научных исследований, от финансовых операций до социальных сетей, сортировка данных играет важную роль в обеспечении оперативного и точного анализа информации.

\par Целью данной лабораторной работы является изучение и сравнение разных реализаций алгоритма поразрядной сортировки (для чисел типа int) -- последовательной реализации и параллельной. Параллельная обработка данных -- это способ ускорения выполнения вычислительных задач за счет распределения работы между несколькими исполнителями (процессорами, ядрами, потоками и т.д.), которые могут выполняться одновременно и независимо друг от друга.

\par Параллельная обработка данных для поразрядной сортировки заключается в том, что исходный массив разбивается на несколько подмассивов, которые сортируются независимо друг от друга на разных потоках. Затем эти подмассивы объединяются в один массив с помощью генетического алгоритма Слияние Бэтчера и как итог, получается, полностью отсортированный массив.

\newpage

% Постановка задачи
\section*{Постановка задачи}
\addcontentsline{toc}{section}{Постановка задачи}
\par В рамках данной работы нужно выполнить следующие задачи:
\begin{enumerate}
    \item Реализовать две версии алгоритма поразрядной сортировки (типа int) со слиянием Бэтчера:
    \begin{itemize}
    \item Последовательная реализация. Алгоритм выполняется в одном потоке. Данная реализация существует для сравнения производительности и корректности работы параллельных реализаций.
    \item Параллельная реализация алгоритма с применением технологии MPI. MPI -- библиотека от компании Microsoft, которая позволяет писать параллельные программы на C, C++, Fortran.
    \end{itemize}
    \item Сравнить время работы разных реализаций алгоритма на различных наборах данных (массив случайных чисел, упорядоченный массив и т.д.).
    \item Сделать выводы по полученным данным.
\end{enumerate}
\newpage

% Описание алгоритма
\section*{Описание алгоритма}
\addcontentsline{toc}{section}{Описание алгоритма}
\par Алгоритм Radix (или сортировка по разрядам) представляет собой метод сортировки целых чисел. Он основан на идее последовательной сортировки чисел по разрядам, начиная с младших и заканчивая старшими. Этот алгоритм может применяться как к десятичным, так и к двоичным числам.

\par Общий шаговый алгоритм Radix:

\begin{enumerate}
    \item Определение количества разрядов: Найдите максимальное число в списке и определите количество его разрядов.
    \item Разбиение по разрядам: Начните с самого младшего разряда и рассортируйте числа по этому разряду. Затем перейдите к следующему более старшему разряду и повторите процесс.
    \item Сортировка по разрядам: Для каждого разряда используйте стабильную сортировку (например, сортировку подсчетом) для упорядочивания чисел.
    \item Повторение процесса: Продолжайте шаги 2-3 для всех разрядов, пока не достигнете самого старшего разряда.
    \item Получение отсортированного списка: После завершения сортировки по всем разрядам вы получите отсортированный список чисел.
\end{enumerate}

\par Преимущество алгоритма Radix в том, что он может быть эффективен для сортировки чисел с фиксированным количеством разрядов. Однако стоимость сортировки по разрядам зависит от количества разрядов и диапазона чисел.
\par Алгоритм Radix может быть реализован для сортировки чисел в различных системах счисления (десятичной, двоичной и т. д.).

\newpage

% Описание схемы распараллеливания
\section*{Описание схемы распараллеливания}

\par В начале вектор данных разделяется на части. Каждая часть сортируется параллельно при помощи алгоритма поразрядной сортировки. Затем отсортированные части объединяются с помощью функции merge, которая реализует алгоритм слияния по генетической схеме.
\par В результате все части сливаются в один вектор.
\newpage

% Описание программной реализации
\section*{Описание программной реализации}
\addcontentsline{toc}{section}{Описание программной реализации}

\par Программа состоит из заголовочного файла \emph{radix\_batcher.h} и двух файлов исходного кода \emph{radix\_batcher.cpp} и \emph{main.cpp}.

\subsection*{Функция getRandomVector} 
Функция принимает в качестве параметров: размер вектора. В ней создаётся вектор конкретного размера и заполяется числа типа double. Функция возвращает созданный вектор.

\subsection*{Функция merge}
\par Функция реализует алгоритм простого слияния двух отсортированных векторов в один тоже отсортированный вектор.
Создаётся вектор, размер которого равен сумме размеров двух входных векторов.
\par После чего в цикле for, который продолжается до достижения значения счётчика значения размера созданного вектора, объединяем массивы, сравнивания элементы каждого между собой и добавляем в результирующий вектор элемент того вектора, значения которого меньше и увеличиваем индекс того вектора, чьё значение добавил в результирующий массив. Если мы пришли к концу одного из векторов, то просто записываем в результирующий массив оставшиеся элементы другого вектора.

\subsection*{Функция oddEvenMerge} 
\par Функция сливает два вектора в один по четно-нечетной схеме.

\subsection*{Функция createCounters} 
Функция создает счетчики для алгоритма Radix.

\subsection*{Функция radixPass} 
\par Функция реализует алгоритм сортировки подсчётом, что требуется для поразрядной сортировки. На вход функции в качестве параметров поступают 2 указателя типа double (входной и выходной массивы) и длина (размер массивов).
Функция возвращает void.

\subsection*{Функция floatRadixLastPass} 
\par Функция реализует алгоритм сортировки подсчётом с некоторыми модернизациями для последнего шага в поразрядной сортировки. На вход функции в качестве параметров поступают 2 указателя типа double (входной и выходной массивы) и длина (размер массивов). функция возвращает значение типа bool.

\subsection*{Функция RadixSortParallel} 
\par Функция запускает Radix по параллельной схеме.

\newpage

% Подтверждение корректности
\section*{Подтверждение корректности}
\addcontentsline{toc}{section}{Подтверждение корректности}
\par Для подтверждения корректности работы программы на фрэймфорке Google Test были написаны 5 тестов для каждой реализации. В каждом тесте запускается умножение матриц разных размеров с разным наполнением.
\par В последовательной реализации в каждом тесте мы просто запускаем алгоритм и сравниваем полученный результат с контрольным. В параллельных версиях мы запускаем оба алгоритма, последовательный и параллельный, и сравниваем результат работы каждого с контрольным значением.
\newpage

% Результаты экспериментов
\section*{Результаты экспериментов}
\addcontentsline{toc}{section}{Результаты экспериментов}
Для проведения экспериментов по вычислению эффективности работы разных реализаций программы использовалась система со следующей конфигурацией:
\begin{itemize}
\item Процессор: AMD Ryzen 5600G, ядер: 6, потоков: 12;
\item Оперативная память: 24 ГБ (DDR4), 3600 МГц;
\item Операционная система: Windows 10 LTSC.
\end{itemize}

\par Результаты экспериментов:
\begin{table}[!h]
\centering
\begin{tabular}{| c | c | c |}
\hline
Версия & Время работы (сек.) & Ускорение (раз) \\
\hline
Последовательный        & 0.0054642        & -         \\
MPI        & 0.0029603        & 1.84          \\
\hline
\end{tabular}
\caption{Результаты экспериментов}
\end{table}

\newpage

% Заключение
\section*{Заключение}
\addcontentsline{toc}{section}{Заключение}
Таким образом, в рамках данной лабораторной работы были разработаны последовательный и параллельный алгоритмы поразрядной сортировки подсчетом со слиянием Бэтчера. Проведенные тесты показали корректность реализованной программы, а проведенные эксперименты доказали эффективность распараллеливания этого алгоритма.
\newpage

% Литература
\section*{Литература}
\addcontentsline{toc}{section}{Литература}
\begin{enumerate}
\item IFMO - Электронный ресурс. URL: \newline \url{http://aco.ifmo.ru/el_books/numerical_methods/lectures/glava2_1.html}
\item Habr - Электронный ресурс. URL: \newline \url{https://habr.com/ru/post/479202/}
\item Educative - Электронный ресурс. URL: \newline \url{https://www.educative.io/blog/modern-multithreading-and-concurrency-in-cpp}
\item А.В. Сысоев, И.Б. Мееров, А.А. Сиднев «Средства разработки параллельных программ для систем с общей памятью. Библиотека Intel Threading Building Blocks». Нижний Новгород, 2007, 128 с. 
\item А.В. Сысоев, И.Б. Мееров, А.Н. Свистунов, А.Л. Курылев, А.В. Сенин, А.В. Шишков, К.В. Корняков, А.А. Сиднев «Параллельное программирование в системах с общей
памятью. Инструментальная поддержка». Нижний Новгород, 2007, 110 с. 
\end{enumerate}


\newpage

\section*{Приложение}
\addcontentsline{toc}{section}{Приложение}

\begin{lstlisting}[language=C++]
// Copyright 2023 Derun Andrew
#ifndef TASKS_TASK_3_DERUN_A_RADIX_BATCHER_RADIX_BATCHER_H_
#define TASKS_TASK_3_DERUN_A_RADIX_BATCHER_RADIX_BATCHER_H_

#include <cstdint>
#include <cstring>
#include <utility>
#include <vector>

std::vector<double> getRandomVector(int sz);
void oddEvenMerge(std::vector<double> *arr, int n, int lo = 0, int r = 1);
std::vector<double> RadixSortParallel(std::vector<double> arr, int size);
std::vector<double> merge(std::vector<std::vector<double>> vectorOfVal);

template <class T>
void createCounters(T *data, uint64_t *counters, uint64_t N) {
  memset(counters, 0, 256 * sizeof(T) * sizeof(uint64_t));

  uint8_t *bp = reinterpret_cast<uint8_t *>(data);
  uint8_t *dataEnd = reinterpret_cast<uint8_t *>(data + N);
  uint16_t i;

  while (bp != dataEnd) {
    for (i = 0; i < sizeof(T); i++) counters[256 * i + *bp++]++;
  }
}

template <class T>
void radixPass(uint16_t Offset, uint64_t N, T *source, T *dest,
               uint64_t *count) {
  T *sp;
  uint64_t s, c, i, *cp;
  uint8_t *bp;

  s = 0;
  cp = count;
  for (i = 256; i > 0; --i, ++cp) {
    c = *cp;
    *cp = s;
    s += c;
  }

  bp = reinterpret_cast<uint8_t *>(source) + Offset;
  sp = source;
  for (i = N; i > 0; --i, bp += sizeof(T), ++sp) {
    cp = count + *bp;
    dest[*cp] = *sp;
    ++(*cp);
  }
}

template <class T>
void floatRadixLastPass(uint16_t Offset, uint64_t N, T *source, T *dest,
                        uint64_t *count) {
  T *sp;
  uint64_t s, c, i, *cp;
  uint8_t *bp;

  uint64_t numNeg = 0;
  for (i = 128; i < 256; i++) numNeg += count[i];

  s = numNeg;
  cp = count;
  for (i = 0; i < 128; ++i, ++cp) {
    c = *cp;
    *cp = s;
    s += c;
  }

  s = count[255] = 0;
  cp = count + 254;
  for (i = 254; i >= 128; --i, --cp) {
    *cp += s;
    s = *cp;
  }

  bp = reinterpret_cast<uint8_t *>(source) + Offset;
  sp = source;
  for (i = N; i > 0; --i, bp += sizeof(T), ++sp) {
    cp = count + *bp;
    if (*bp < 128)
      dest[(*cp)++] = *sp;
    else
      dest[--(*cp)] = *sp;
  }
}

template <class T>
void floatRadixSort(T **in, uint64_t N) {
  T *out = new T[N];
  uint16_t i;

  uint64_t *counters = new uint64_t[sizeof(T) * 256], *count;
  createCounters((*in), counters, N);

  for (i = 0; i < sizeof(T) - 1; i++) {
    count = counters + 256 * i;
    if (count[0] == N) continue;
    radixPass(i, N, (*in), out, count);
    std::swap((*in), out);
  }
  count = counters + 256 * i;
  floatRadixLastPass(i, N, (*in), out, count);

  delete (*in);
  (*in) = out;
  delete[] counters;
}

#endif  // TASKS_TASK_3_DERUN_A_RADIX_BATCHER_RADIX_BATCHER_H_
\end{lstlisting}

\begin{lstlisting}[language=C++]
// Copyright 2023 Derun Andrew
#include "task_3/derun_a_radix_batcher/radix_batcher.h"

#include <mpi.h>

#include <algorithm>
#include <cmath>
#include <ctime>
#include <random>
#include <utility>
#include <vector>

std::vector<double> getRandomVector(const int size) {
  std::mt19937 gen(time(0));
  std::uniform_real_distribution<> urd(0, 300);
  std::vector<double> vector(size);
  for (double& value : vector) value = urd(gen);

  return vector;
}

void oddEvenMerge(std::vector<double>* arr, int n, int lo, int r) {
  int m = r * 2;

  if (m < n) {
    oddEvenMerge(arr, n, lo, m);
    oddEvenMerge(arr, n, lo + r, m);

    for (int i = lo + r; i + r < lo + n; i += m) {
      if ((*arr)[i] > (*arr)[i + r]) std::swap((*arr)[i], (*arr)[i + r]);
    }
  } else {
    if ((*arr)[lo] > (*arr)[lo + r]) {
      std::swap((*arr)[lo], (*arr)[lo + r]);
    }
  }
}

std::vector<double> merge(std::vector<std::vector<double>> vector_array) {
  while (vector_array.size() != 1) {
    for (int i = 0; i < static_cast<int>(vector_array.size() - 1); i++) {
      auto temp = vector_array[i];
      temp.insert(temp.end(), vector_array[i + 1].begin(),
                  vector_array[i + 1].end());
      oddEvenMerge(&temp, temp.size());
      vector_array[i] = temp;
      vector_array.erase(vector_array.begin() + i + 1);
    }
  }
  return vector_array[0];
}

std::vector<double> RadixSortParallel(std::vector<double> vec, int size) {
  int nproc, rank;
  MPI_Comm_size(MPI_COMM_WORLD, &nproc);
  MPI_Comm_rank(MPI_COMM_WORLD, &rank);

  int delta = size / nproc;
  std::vector<double> local_vec, global_vec;

  local_vec.resize(delta);

  MPI_Scatter(vec.data(), delta, MPI_DOUBLE, local_vec.data(), delta,
              MPI_DOUBLE, 0, MPI_COMM_WORLD);
  double* data = local_vec.data();
  floatRadixSort<double>(&data, delta);

  if (rank != 0) {
    MPI_Send(local_vec.data(), delta, MPI_DOUBLE, 0, 0, MPI_COMM_WORLD);
  } else {
    std::vector<std::vector<double>> vec_array;
    vec_array.push_back(local_vec);

    for (int i = 1; i < nproc; ++i) {
      MPI_Recv(local_vec.data(), delta, MPI_DOUBLE, i, 0, MPI_COMM_WORLD,
               MPI_STATUSES_IGNORE);
      vec_array.push_back(local_vec);
    }

    global_vec = merge(vec_array);
  }

  return global_vec;
}

\end{lstlisting}

\begin{lstlisting}[language=C++]
// Copyright 2023 Derun Andrew
#include <gtest/gtest.h>
#include <mpi.h>

#include <cmath>
#include <vector>

#include "task_3/derun_a_radix_batcher/radix_batcher.h"

TEST(RADIX_BATCHER, TEST_1) {
  int rank, size;
  MPI_Comm_rank(MPI_COMM_WORLD, &rank);
  MPI_Comm_size(MPI_COMM_WORLD, &size);
  if (log2(size) == static_cast<int>(log2(size))) {
    if (rank == 0) {
      auto vec = getRandomVector(128);
      double* data = vec.data();
      floatRadixSort<double>(&data, 128);
      for (int i = 0; i < static_cast<int>(vec.size()) - 1; i++) {
        EXPECT_TRUE(vec[i] <= vec[i + 1]);
      }
    }
  }
}

TEST(RADIX_BATCHER, TEST_2) {
  int rank, size;
  MPI_Comm_rank(MPI_COMM_WORLD, &rank);
  MPI_Comm_size(MPI_COMM_WORLD, &size);
  if (log2(size) == static_cast<int>(log2(size))) {
    std::vector<double> vec;
    if (rank == 0) {
      vec = getRandomVector(128);
    }
    vec = RadixSortParallel(vec, 128);
    if (rank == 0) {
      for (int i = 0; i < static_cast<int>(vec.size()) - 1; i++) {
        EXPECT_TRUE(vec[i] <= vec[i + 1]);
      }
    }
  }
}

TEST(RADIX_BATCHER, TEST_3) {
  int rank, size;
  MPI_Comm_rank(MPI_COMM_WORLD, &rank);
  MPI_Comm_size(MPI_COMM_WORLD, &size);
  if (log2(size) == static_cast<int>(log2(size))) {
    std::vector<double> vec;
    if (rank == 0) {
      vec = getRandomVector(pow(2, 8));
    }
    vec = RadixSortParallel(vec, pow(2, 8));
    if (rank == 0) {
      for (int i = 0; i < static_cast<int>(vec.size()) - 1; i++) {
        EXPECT_TRUE(vec[i] <= vec[i + 1]);
      }
    }
  }
}

TEST(RADIX_BATCHER, TEST_4) {
  int rank, size;
  MPI_Comm_rank(MPI_COMM_WORLD, &rank);
  MPI_Comm_size(MPI_COMM_WORLD, &size);
  if (log2(size) == static_cast<int>(log2(size))) {
    std::vector<double> vec;
    if (rank == 0) {
      vec = getRandomVector(pow(2, 9));
    }
    vec = RadixSortParallel(vec, pow(2, 9));
    if (rank == 0) {
      for (int i = 0; i < static_cast<int>(vec.size()) - 1; i++) {
        EXPECT_TRUE(vec[i] <= vec[i + 1]);
      }
    }
  }
}

TEST(RADIX_BATCHER, TEST_5) {
  int rank, size;
  MPI_Comm_rank(MPI_COMM_WORLD, &rank);
  MPI_Comm_size(MPI_COMM_WORLD, &size);
  if (log2(size) == static_cast<int>(log2(size))) {
    std::vector<double> vec, vec_s;
    double start, end;
    if (rank == 0) {
      vec = getRandomVector(pow(2, 8));
      vec_s = vec;
    }
    start = MPI_Wtime();
    vec = RadixSortParallel(vec, pow(2, 8));
    end = MPI_Wtime();
    if (rank == 0) {
      double stime = end - start;
      double* data = vec_s.data();
      start = MPI_Wtime();
      floatRadixSort<double>(&data, pow(2, 8));
      end = MPI_Wtime();
      double ptime = end - start;
      for (int i = 0; i < static_cast<int>(vec.size()); i++) {
        EXPECT_DOUBLE_EQ(vec[i], vec_s[i]);
        // std::cout << vec[i] << " " << vec_s[i] << std::endl;
      }
      std::cout << "P time " << ptime << std::endl;
      std::cout << "S time " << stime << std::endl;
      std::cout << "Speedup " << stime / ptime << std::endl;
    }
  }
}

int main(int argc, char** argv) {
  int result_code = 0;

  ::testing::InitGoogleTest(&argc, argv);
  ::testing::TestEventListeners& listeners =
      ::testing::UnitTest::GetInstance()->listeners();

  if (MPI_Init(&argc, &argv) != MPI_SUCCESS) MPI_Abort(MPI_COMM_WORLD, -1);
  result_code = RUN_ALL_TESTS();
  MPI_Finalize();

  return result_code;
}

\end{lstlisting}

\end{document}